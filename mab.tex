% This file was converted to LaTeX by Writer2LaTeX ver. 1.6.1
% see http://writer2latex.sourceforge.net for more info
\documentclass[letterpaper,twocolumn,9pt]{article}
\usepackage[ascii]{inputenc}
\usepackage{amsmath}
\usepackage{amssymb,amsfonts,textcomp}
\usepackage[T1]{fontenc}
\usepackage[english]{babel}
\usepackage{color}
\usepackage{multicol}
\usepackage{array}
\usepackage{hhline}
\usepackage{hyperref}
\hypersetup{pdftex, colorlinks=true, linkcolor=blue, citecolor=blue, filecolor=blue, urlcolor=blue, pdftitle=, pdfauthor=Alex Towell, pdfsubject=, pdfkeywords=}
\usepackage[pdftex]{graphicx}
\providecommand\textsubscript[1]{\ensuremath{{}_{\text{#1}}}}
% Page layout (geometry)
\setlength\voffset{-1in}
\setlength\hoffset{-1in}
\setlength\topmargin{0.5in}
\setlength\oddsidemargin{1in}
\setlength\textheight{9.0in}
\setlength\textwidth{6.5in}
\setlength\footskip{0.5in}
\setlength\headheight{0.5in}
\setlength\headsep{0cm}
% Footnote rule
\setlength{\skip\footins}{1.1777999mm}
\renewcommand\footnoterule{\vspace*{-0.007in}\setlength\leftskip{0pt}\setlength\rightskip{0pt plus 1fil}\noindent\textcolor{black}{\rule{0.33\columnwidth}{0.007in}}\vspace*{1mm}}
% Pages styles
\makeatletter
\newcommand\ps@MP{
  \renewcommand\@oddhead{}
  \renewcommand\@evenhead{\@oddhead}
  \renewcommand\@oddfoot{\thepage{}}
  \renewcommand\@evenfoot{\@oddfoot}
  \renewcommand\thepage{\arabic{page}}
}
\makeatother
\pagestyle{MP}
% List styles
\newcounter{saveenum}
\newcommand\liststyleLFOxx{%
\renewcommand\theenumi{\roman{enumi}}
\renewcommand\labelenumi{(\theenumi)}
\renewcommand\labelitemi{o}
\renewcommand\labelitemii{${\blacksquare}$}
\renewcommand\labelitemiii{{\textbullet}}
}
\newcommand\liststyleLFOxxvii{%
\renewcommand\labelitemi{{\textbullet}}
\renewcommand\labelitemii{o}
\renewcommand\labelitemiii{${\blacksquare}$}
\renewcommand\labelitemiv{{\textbullet}}
}
\newcommand\liststyleLFOxxv{%
\renewcommand\labelitemi{{\textbullet}}
\renewcommand\labelitemii{o}
\renewcommand\labelitemiii{${\blacksquare}$}
\renewcommand\labelitemiv{{\textbullet}}
}
\newcommand\liststyleLFOxii{%
\renewcommand\labelitemi{[F096?]}
\renewcommand\labelitemii{[F096?]}
\renewcommand\labelitemiii{[F096?]}
\renewcommand\labelitemiv{[F096?]}
}
\title{
    Estimating Confidentiality of Encrypted Searches using a Moving Average 
    Bootstrap Method
}
\author{Alexander Towell and Hiroshi Fujinoki}
\date{2016-12-30}
\begin{document}
\begin{abstract}
    Frequency attacks are a type of attacks adversaries perform to 
    encrypted searches. Adversaries guess the meaning of the encrypted words by 
    observing a large number of encrypted words and map the encrypted words to 
    guessed plain-text words using their known histogram. This paper\ applies 
    an approach of resilience engineering in studying how effective encrypted 
    searches will be for protecting users' privacy when they perform oblivious 
    searches, i.e., searches\ on their encrypted data. For estimating the 
    number of encrypted words an adversary needs to observe before the 
    adversary correctly guesses a  certain percentage of the observed words 
    with\ a certain confidence, we developed and evaluated Moving Average 
    Bootstrap (MAB) method. In the experiments\ we performed, we observed that 
    MAB method reliably and accurately estimated in 5\% of the time a defender 
    would have to wait before the defender could calculate the estimator.
\end{abstract}

Department of Computer Science
Southern Illinois University Edwardsville
Edwardsville, IL 62026-1656\newline
+1 618 650 3727
{atowell,hfujino}@siue.edu


Categories and Subject Descriptors

\textrm{\textcolor{black}{H.3.3 
[}}\textrm{\textbf{\textcolor{black}{Information 
Systems}}}\textrm{\textcolor{black}{]:
Information Search and Retrieval - Information filtering; Relevance feedback; Search process;\ }}

\textrm{\textcolor{black}{E.2 [}}\textrm{\textbf{\textcolor{black}{Data}}}\textrm{\textcolor{black}{]: Data
Storage\ Presentations\ {}- Object representation;}}


\bigskip

{\rmfamily\bfseries\color{black}
Keywords}

{\rmfamily\color{black}
Encrypted searches,\ known-plaintext attacks,\ oblivious searches,\ bootstrap method,\ information retrieval,
information security,\ resilience engineering}


\bigskip

{\rmfamily\bfseries\color{black}
1.\ Introduction}


\bigskip

{\rmfamily\color{black}
As electric information systems have been our essential infrastructure in our society, the\ risk
of\ unauthorized\ information\ leaks has been rising. \ Such information leaks have\ become a serious issue since
recent popular adoptions of cloud computing\ could end up with social turmoil. \ For example, malicious activities
conducted by the people we\ most trust;\ system administrators\ at cloud providers; have been\ threats that may prevent
further adoptions, or even\ the death\ of,\ the\ network-based electronic information systems,\ if we do not take
proper actions against such crimes [1, 2, 3].}

{\rmfamily\color{black}
The essential challenge here is a trade-off problem between the privacy and the\ usability\ in the data we deposit to
remote systems, such as cloud servers.\ \ To resolve this trade-off problem, researchers invented a\ concept, called
``encrypted search''.}

{\rmfamily\color{black}
A term, ``encrypted\ search'', means a concept that lets\ authorized\ users\ to investigate presence of specific words
or phrases in\ a target data set, such as an encrypted document or a database table\ where stored\ data\ is
encrypted\ [4, 5, 6, 7, 8], while the contents, especially the meaning of the data,\ are hidden from any\ unauthorized
personnel, including the system administrators of a\ cloud server.}

\textrm{\textcolor{black}{Encrypted\ search is expected to bring us tremendous benefits in information security.\ \ For
example, encrypted search\ will\ enable\ us\ to perform\ }}\textrm{\textit{\textcolor{black}{oblivious
searches}}}\textrm{\textcolor{black}{\ in remote database systems. \ Users would be able to perform specific
searches\ in remote database systems or search engines,\ such as Google, without anyone, including their system
administrators who usually have full access to all the queries issued by their users, knowing what words or phrases are
searched.}}

\textrm{\textcolor{black}{Despite its potentials,}}\textrm{\ no\ encrypted
search}\textrm{\textit{\ }}\textrm{scheme\ theoretically\ guarantees\ perfect\ confidentiality.\ \ There are many ways
information may be leaked.\ For example,\ encrypted search\ queries use a substitution cipher where plain text words
are substituted with encrypted\ }\textrm{\textcolor{black}{counterparts\ in queries.\ \ This causes a major
vulnerability in encrypted searches.}}

\textrm{\textcolor{black}{A major known threat to encrypted search is\ frequency analysis\ (aka, ``spectral
analysis'')\ attacks\ [9]. \ If\ adversaries\ can estimate the relative frequency\ of plain text words, then the most
frequently occurring encrypted}}\textrm{\ word probably maps to the most frequently occurring plain text word, which is
known as ``known-plaintext attacks''.\ Adversaries look\ for a mapping that maximizes the likelihood of
seeing\ }\textrm{those encrypted words under the estimated distribution.\ }

\textrm{Adversaries\ may\ know that an encrypted document is for a specific purpose and, therefore, they\ may\ predict
some\ }\textrm{\textcolor{black}{terms or phrases to appear quite often in the document. \ According to\ Piantadosi,
the text frequency distribution\ of\ words in\ most\ documents\ follows Zipf distribution [10]. \ Adversaries utilize
the expected frequency for guessing their plain text counterparts.\ \ These systems are at a risk of information leaks.
\ }}

\textrm{\textcolor{black}{We applied the bootstrap method in quantifying the level of accuracy adversaries can achieve,
while we face two major uncertainty factors:\ [F081?]\ as we describe in Section 3,\ since the actual\ accuracy in
adversary's\ will\ fluctuate\ in unpredictable ways, it is not easy for the defender side to determine the level of
accuracy adversaries may have achieved\ for each particular system, and\ [F082?]\ observing a large number of samples
(i.e., encrypted words) for accurately estimating\ }}\textrm{\textit{\textcolor{black}{N}}}\textrm{\textcolor{black}{*
takes time for defenders.}}

{\rmfamily\color{black}
To cope with the issues described above, we developed and evaluated a new\ method\ that accurately and efficiently
estimates the vulnerability by adversaries (how many encrypted words an adversary needs to observe to correctly map a
certain number of encrypted words to those in plain\ text and how likely an adversary can achieve the goal).}

{\rmfamily\color{black}
The rest of this paper is organized as follows. \ In Section 2, we review the existing effort to encrypted searches.
\ There has not been much work that\ quantitatively\ analyzes the conditions for information leaks by frequency
attacks, such as the number of encrypted words an adversary needs to observe for a certain accuracy and how likely it
happens. \ In Section 3,\ we introduce our moving average bootstrap\ (MAB)\ method to accurately and efficiently
estimate the achievable accuracy for\ frequency\ attacks performed by adversaries. In the attacks,
adversaries'\ accuracy often fluctuates and adversaries\ may observe potentially a large number of encrypted words,
which is theoretically infinity\ especially\ if\ adversaries\ are\ system\ administrators. \ In Section 4, we present
our performance evaluations on\ the MAB\ method. Section 5\ summarizes our contributions and planned future work,
followed by the selected references.}


\bigskip

{\rmfamily\bfseries\color{black}
2.\ Existing Work}


\bigskip

{\rmfamily\color{black}
Boneh\ proposed\ a method to let a third party to perform searches over encrypted e-mail messages, called public-key
encryption with keyword search (PEKS) [4]. \ Boneh designed\ PEKS\ in such a way\ that e-mail messages are encrypted
by\ the public key\ of\ an e-mail receiver,\ while a third party, such as an e-mail server, to perform search for a
particular word (e.g., ``urgent'')\ in each encrypted message\ without all the\ raw\ contents in\ the\ encrypted
e-mail\ exposed to the third party.\ \ The core of this\ method is trapdoors, which are\ a hash value of a given
word\ in e-mails. \ Each e-mail receiver creates trapdoors,\ one for each target word\ and trapdoors are included in
each encrypted e-mail message for searches on the encrypted e-mail messages.}

{\rmfamily\color{black}
Li extended this concept to allow third parties to perform encrypted searches that allows deviations from exact matching
to target encrypted words by\ enumerating\ multiple trapdoors, one for each expected deviation [5].\ Cao [6], Sun [7],
and Kamara [8]\ proposed to apply encrypted search to enhancing security in cloud computing.}

{\rmfamily\color{black}
Despite the potentials in\ the\ encrypted search schemes,\ risk of information leaks through guessing the searched words
has been identified [11,\ 12,\ 13]. Byun [11],\ Yau [12], and Jeong [13]\ technically demonstrated that anyone who has
access to encrypted data possibly map them to\ their plain\ text counterparts.}

{\rmfamily\color{black}
Use of secure communication channels (e.g., SSL) will be effective in hiding the trapdoors in queries submitted by
legitimate users\ from external adversaries, but use of secure communication channels\ still\ can not prevent frequency
attacks from internal adversaries, such as malicious administrators, assuming that they can intercept trapdoors\ within
a local host computer, by installing illegal capturing tool or by tampering executables.}

{\rmfamily\color{black}
Despite the threat from\ frequency\ attacks, there has not been much\ work\ that\ delves\ into quantified analyses on
the conditions for when such information leaks\ exceed\ a\ tolerable\ risk\ level\ under\ various conditions.\ \ Rivain
proposed\ a multivariate Gaussian random variable method to estimate the success rate in discovering secret
keys\ under\ side-channel attacks\ [14].\ Thillard\ proposed use of ``confidence'' for evaluating the effectiveness
in\ side-channel attacks\ [15].\ \ Rivain and Thillard's both\ proposed a solution against\ correlation attacks, but
not\ against frequency attacks. \ Correlation attacks are different from frequency attacks in that\ adversaries
discover the encryption keys\ to deduce the plain texts\ in\ the former, while the latter\ induces\ the plain texts
directly from the observed encrypted words without discovering their encryption keys.}


\bigskip

{\rmfamily\bfseries\color{black}
3.\ \ Moving Average Bootstrap (MAB) Method}


\bigskip

{\rmfamily\color{black}
We applied the bootstrap method to efficiently estimate how vulnerable encrypted searches will
be\ against\ frequency\ attacks\ especially when adversaries are expected to\ observe\ a large number of\ encrypted
words\ for long period of time.\ Bootstrap is a method for statistical estimation\ of various properties on population,
such as variance, by calculating\ the properties\ using\ only\ the limited number of samples\ that follow\ a
distributed approximation [16].}

{\rmfamily\color{black}
We applied the bootstrap method for estimating the vulnerabilities to encrypted searches from adversaries\ under the
following\ four\ assumptions.}

\liststyleLFOxx
\setcounter{saveenum}{\value{enumi}}
\begin{enumerate}
\setcounter{enumi}{\value{saveenum}}
\item {\rmfamily\color{black}
Adversaries know, in advance, that an encrypted document is for a specific purpose and, therefore, they
predict\ histograms of the\ terms or phrases (thus, their objective is mainly\ to discover\ who searched for what
terms).}
\item {\rmfamily\color{black}
Adversaries have access to every encrypted search query submitted by legitimate users.}
\item {\rmfamily\color{black}
The encrypted search keys are\ encrypted using word\ cypher\ instead of block cypher to allow search engines to perform
uni-gram searches for each search key.\ \ This class of attacks is known as\ cipher text-only attack\ (COA).}
\item {\rmfamily\color{black}
Adversaries are stand-alone.\ They do not\ coordinate their\ frequency\ attacks. \ Thus, each of them will see different
sequence of samples\ (i.e., encrypted words)\ and they do not\ share their histograms of their samples\ with other
adversaries.\ \ This assumption is for simplifying our analyses on vulnerability on encrypted search, to be relaxed in
our future work.}
\end{enumerate}
\textrm{\textcolor{black}{To quantify the vulnerability to\ frequency\ attacks, we defined a\ term, ``adversary's
accuracy{}''. \ Adversary accuracy\ (}}\textrm{\textit{\textcolor{black}{p*}}}\textrm{\textcolor{black}{)\ is the ratio
of the number of the correctly mapped text search keys\ to that of the\ total\ encrypted search keys\ observed by an
adversary, which is defined as:\ \ }}

\textrm{\textcolor{black}{adversary
accuracy\ (}}\textrm{\textit{\textcolor{black}{p*}}}\textrm{\textcolor{black}{)\ =\ ${\delta}$/}}\textrm{\textit{\textcolor{black}{n}}}\textrm{\textcolor{black}{\ \ \ \ \ \ \ \ \ (1)}}

{\rmfamily\color{black}
where:}

\textrm{\textcolor{black}{${\delta}$\ = the number of the correctly mapped text search keys\ by an adversary}}

\textrm{\textit{\textcolor{black}{n}}}\textrm{\textcolor{black}{\ =\ the number of encrypted search keys\ observed
by\ an adversary in a session of his\ frequency\ attack}}

\textrm{\textcolor{black}{For example, for the\ first\ sequence of encrypted words (the words in the following
examples\ are shown as non-encrypted words to make its meaning clear\ to the audience) in
Figure\ 1\ (i),\ }}\textrm{\textit{\textcolor{black}{n}}}\textrm{\textcolor{black}{\ =\ 8 (a duplicated word,
``world''\ is\ counted twice).}}

  [Warning: Image ignored] % Unhandled or unsupported graphics:
%\includegraphics[width=3in,height=0.67708in]{mab-img001.emf}
 

\textrm{\textbf{\textcolor{black}{Figure\ 1}}}\textrm{\textcolor{black}{: two examples of sequences of words}}

\textrm{\textcolor{black}{Assuming that the following words\ in the first sequence\ (i)\ are correctly mapped to their
plain words by an adversary: ``hello'', ``world'', ``Alex'' and ``wonderful'',\ ${\delta}$\ =
4\ and\ }}\textrm{\textit{\textcolor{black}{p}}}\textrm{\textcolor{black}{* = 0.5 (= 4/8). \ If another sequence of
encrypted words\ (Figure\ 1\ (ii))\ follows the previous\ sequence\ (i), and if the following encrypted words are
correctly mapped\ to their plain texts by the same adversary: ``Illinois'', ``hello'', and
``Alex'',\ }}\textrm{\textit{\textcolor{black}{n}}}\textrm{\textcolor{black}{\ = 17\ and\ ${\delta}$\ = 7, resulting
in\ }}\textrm{\textit{\textcolor{black}{p}}}\textrm{\textcolor{black}{*\ ${\cong}$\ 0.412 (${\cong}$\ 7/17).}}

{\rmfamily\color{black}
The primary challenge\ in\ accurately estimating the achieved accuracy is fluctuations in adversary accuracy.\ The
inherent fluctuations in adversary accuracy will make\ accurately estimating the achievable accuracy difficult, which
will prohibit\ defenders\ from designing information systems with an expected level of resilience
against\ frequency\ attacks on encrypted searches.\ \ Thus, we\ need\ an efficient\ measure of accuracy
that\ is\ unaffected by such fluctuations.}

{\rmfamily\color{black}
Answering the above questions is important especially for\ defenders, since having a systematic method to accurately and
efficiently estimate the achievable accuracy by adversaries in given conditions would allow\ them\ to design a system
that has a specific level of resilience against\ frequency\ attacks on encrypted searches. \ Such a method will provide
a foundation for the safety of our information systems in asking security-related critical questions, such
as,\ {}``would you be ok with someone only being able to read 70\% of your private journal\ with\ a\ 1\%\ chance?''}

{\rmfamily\color{black}
To cope with the expected fluctuations in adversary accuracy, we developed a new estimation method\ called\ moving
average bootstrap\ (MAB) method. \ The proposed estimation\ method\ performs smoothing out of fluctuating adversary
accuracy using the bootstrap method. \ The new method\ smooths\ out\ the fluctuation curve as the measure for the
adversary's current accuracy using a moving average.}

\textrm{\textcolor{black}{The primary concern in the proposed estimation\ method\ is to answer the following question:
when is the risk too high that the adversary has achieved a certain level of accuracy?\ \ To quantitatively find an
answer for the question, we introduced a metric, the\ estimator for the\ minimum number of encrypted\ words\ an
adversary needs to achieve a given accuracy level
of\ }}\textrm{\textit{\textcolor{black}{p}}}\textrm{\textcolor{black}{*, which is represented
by\ }}\textrm{\textit{\textcolor{black}{N}}}\textrm{\textcolor{black}{*.}}

\textrm{\textcolor{black}{The concept of\ {}``the estimator for\ the minimum number of encrypted\ words\ an adversary
needs to\ observe for achieving\ a given accuracy level
of\ }}\textrm{\textit{\textcolor{black}{p}}}\textrm{\textcolor{black}{*'' for
a}}\textrm{\textit{\textcolor{black}{\ specific\ }}}\textrm{\textcolor{black}{attempt of\ frequency\ attack, which we
represent by ``}}\textrm{\textit{\textcolor{black}{n}}}\textrm{\textcolor{black}{*'' (instead of
``}}\textrm{\textit{\textcolor{black}{N}}}\textrm{\textcolor{black}{*'')\ is visualized in Figure 2. \ Figure 2\ shows
the plots of adversary accuracy\ estimated\ by an adversary. \ Mainly\ due\ to\ }}\textrm{\textcolor{black}{the
randomness\ in each sequence of encrypted words an adversary observes, the adversary's accuracy will fluctuate, causing
some spikes in both up and down ward (we marked one of such upward spikes as ``fluke high point'' in
Figure\ 2),\ making the adversary unsure if he has achieved a target accuracy
level\ (e.g.,\ }}\textrm{\textit{\textcolor{black}{p}}}\textrm{\textcolor{black}{* = 0.36\ was applied to\ this
example).\ \ However, if the adversary's accuracy never goes
below\ }}\textrm{\textit{\textcolor{black}{p}}}\textrm{\textcolor{black}{* = 0.36 after\ a certain number of the
observed encrypted words, the adversary is sure that he has achieved the accuracy of 36\%.\ \ We used a
symbol,\ }}\textrm{\textit{\textcolor{black}{n}}}\textrm{\textcolor{black}{*, to represent the number of the encrypted
words.}}

  [Warning: Image ignored] % Unhandled or unsupported graphics:
%\includegraphics[width=3in,height=2.0625in]{mab-img002.emf}
 

\textrm{\textbf{\textcolor{black}{Figure\ 2}}}\textrm{\textcolor{black}{:\ an example of plots of\ adversary accuracy
(}}\textrm{\textit{\textcolor{black}{p}}}\textrm{\textcolor{black}{*) for\ a session\ of\ frequency\ attack and
fluctuations in the adversary accuracy}}


\bigskip

\textrm{\textcolor{black}{It is expected\ that\ }}\textrm{\textit{\textcolor{black}{n}}}\textrm{\textcolor{black}{*
value will be different for each attempt of\ frequency\ attack.\ If three different adversaries perform their
own\ frequency\ attack to the same target data set,
their\ }}\textrm{\textit{\textcolor{black}{n}}}\textrm{\textcolor{black}{* values for the same accuracy level will be
different even if they observed the same number of encrypted words\ for the same data set\ just because the sequences
of the encrypted words observed in each attempt will be different. Thus, one particular value
of\ }}\textrm{\textit{\textcolor{black}{n}}}\textrm{\textcolor{black}{* will not be reliable enough. \ As a result, we
need to\ efficiently calculate\ {}``the\ estimator for the\ minimum number of encrypted search queries an adversary
needs to achieve a given accuracy level of\ }}\textrm{\textit{\textcolor{black}{p}}}\textrm{\textcolor{black}{*'' with
a reasonably high level of confidence, which is\ denoted
by\ {}``}}\textrm{\textit{\textcolor{black}{N}}}\textrm{\textcolor{black}{*''.\ \ To accurately\ and
efficiently\ estimate\ }}\textrm{\textit{\textcolor{black}{N}}}\textrm{\textcolor{black}{* from a limited number of
observed\ }}\textrm{\textit{\textcolor{black}{n}}}\textrm{\textcolor{black}{*'s,\ while observed adversary accuracies
fluctuate in unpredictable manners,\ we applied the bootstrap method.}}

\textrm{\textcolor{black}{We applied the bootstrap
method\ to\ calculating\ }}\textrm{\textit{\textcolor{black}{N}}}\textrm{\textcolor{black}{*\ when adversaries can
logically perform a\ large number of\ frequency attack\ sessions, each of which lets each adversary to observe a large
number of encrypted words. \ To achieve the goals,\ we designed our\ bootstrap method\ using\ the following
parameters:}}

\liststyleLFOxxvii
\begin{itemize}
\item \textrm{\textit{\textcolor{black}{n}}}\textrm{\textcolor{black}{\ (the number of the encrypted words\ an adversary
observes):\ legitimate users are expected\ to send search keys as encrypted words so that the\ search\ engine\ can look
for matching words over the encrypted words stored in\ a target data set.
\ The\ }}\textrm{\textit{\textcolor{black}{n}}}\textrm{\textcolor{black}{\ encrypted words each adversary observes are
assumed to be different for each adversary.}}
\end{itemize}
\liststyleLFOxxv
\begin{itemize}
\item \textrm{\textit{\textcolor{black}{p}}}\textrm{\textcolor{black}{*\ (a\ level of accuracy an adversary attempts to
achieve\ (0.0 ${\leq}$\ }}\textrm{\textit{\textcolor{black}{p}}}\textrm{\textcolor{black}{* ${\leq}$ 1.0)):
\ when\ }}\textrm{\textit{\textcolor{black}{p}}}\textrm{\textcolor{black}{* = 0.0, an adversary does
not\ successfully\ map any encrypted word to its correct word in plain text,
while\ }}\textrm{\textit{\textcolor{black}{p}}}\textrm{\textcolor{black}{* = 1.0 means that an adversary successfully
mapped all the encrypted words he observed to the correct words
(thus,\ }}\textrm{\textit{\textcolor{black}{p}}}\textrm{\textcolor{black}{* = 0.5 means the half of the observed
encrypted words are mapped to the correct words in plain text).}}
\item \textrm{\textit{\textcolor{black}{n}}}\textrm{\textcolor{black}{* (the minimum number of encrypted search queries
an adversary needs to observe to achieve a given accuracy level
of\ }}\textrm{\textit{\textcolor{black}{p}}}\textrm{\textcolor{black}{*): it indicates the smallest value
of\ }}\textrm{\textit{\textcolor{black}{n}}}\textrm{\textcolor{black}{\ in such the way that an adversary never goes
below the given accuracy level in an attempt of\ frequency attack\ (see Figure\ 2\ for its visualization). \ \ Since it
is impossible for testing a frequency attack on an infinitely large number
of\ }}\textrm{\textit{\textcolor{black}{n}}}\textrm{\textcolor{black}{, we assumed that
10}}\textrm{\textcolor{black}{\textsuperscript{9}}}\textrm{\textcolor{black}{\ is ``infinity'' if there was no increase
in\ }}\textrm{\textit{\textcolor{black}{p}}}\textrm{\textcolor{black}{*
from\ }}\textrm{\textit{\textcolor{black}{n}}}\textrm{\textcolor{black}{\ =
10}}\textrm{\textcolor{black}{\textsuperscript{9}}}\textrm{\textcolor{black}{\ to\ }}\textrm{\textit{\textcolor{black}{n}}}\textrm{\textcolor{black}{\ =
5\ ${\times}$\ 10}}\textrm{\textcolor{black}{\textsuperscript{9}}}\textrm{\textcolor{black}{.}}
\item \textrm{\textit{\textcolor{black}{${\theta}$}}}\textrm{\textcolor{black}{\ (the probability an adversary achieves
an\ given\ accuracy level of\ }}\textrm{\textit{\textcolor{black}{p}}}\textrm{\textcolor{black}{*\ (0.0
${\leq}$\ }}\textrm{\textit{\textcolor{black}{${\theta}$\ }}}\textrm{\textcolor{black}{${\leq}$
1.0)):\ when\ }}\textrm{\textit{\textcolor{black}{${\theta}$}}}\textrm{\textcolor{black}{\ = 0.5, 50\% of
adversaries,\ each of\ whom\ performed an attempt of\ frequency attack\ to the same\ target data\ set,\ will achieve
the given accuracy\ (i.e., successfully mapped
(}}\textrm{\textit{\textcolor{black}{p}}}\textrm{\textcolor{black}{*\ ${\cdot}$\ 100)\% of his observed encrypted words
to their plain counterparts) by observing\ }}\textrm{\textit{\textcolor{black}{N}}}\textrm{\textcolor{black}{* samples
(i.e., encrypted words).\ \ }}\textrm{\textit{\textcolor{black}{${\theta}$\ }}}\textrm{\textcolor{black}{= 1.0
means\ every adversary achieves the given accuracy level,
while\ }}\textrm{\textit{\textcolor{black}{${\theta}$\ }}}\textrm{\textcolor{black}{= 0.0 means no adversary achieves
the given accuracy level.}}
\item \textrm{\textit{\textcolor{black}{k}}}\textrm{\textcolor{black}{\ (conjectured\ sample size for population set):
it\ is the number of\ frequency attack\ sessions (each of which produced its
own\ }}\textrm{\textit{\textcolor{black}{n}}}\textrm{\textcolor{black}{*) as our best estimation for\ an infinitely
large number\ of\ frequency attack\ sessions.\ \ We justified our selection
of\ }}\textrm{\textit{\textcolor{black}{k}}}\textrm{\textcolor{black}{\ (}}\textrm{\textit{\textcolor{black}{k}}}\textrm{\textcolor{black}{\ =
5,000) as follows. \ As we showed in Figure\ 6\ later in this paper, the 95\% confidence interval for the population
set was [6300, 7550] for\ }}\textrm{\textit{\textcolor{black}{p}}}\textrm{\textcolor{black}{* = 0.30
and\ }}\textrm{\textit{\textcolor{black}{${\theta}$}}}\textrm{\textcolor{black}{\ = 0.50. \ We repeated the experiments
(5,000 samples for our approximation of the population set) 10,000 times, resulting in a total of 50 million samples to
produce the 95\% confidence interval of 1,250 (i.e., 7,550 -- 6,300). \ \ The interval is relatively small for the 50
million samples, where their ratio is 0.000025 (i.e., 1,250/50,000,000).}}
\item \textrm{\textit{\textcolor{black}{m}}}\textrm{\textcolor{black}{\ (the bootstrap sample size):\ it\ is the number
of samples (i.e.,\ }}\textrm{\textit{\textcolor{black}{n}}}\textrm{\textcolor{black}{*{}'s)\ randomly\ selected from
the
set\ }}\textrm{\textcolor{black}{of\ }}\textrm{\textit{\textcolor{black}{k}}}\textrm{\textcolor{black}{\ }}\textrm{\textit{\textcolor{black}{n}}}\textrm{\textcolor{black}{*{}'s
(}}\textrm{\textit{\textcolor{black}{m}}}\textrm{\textcolor{black}{\ $<$$<$\ }}\textrm{\textit{\textcolor{black}{k}}}\textrm{\textcolor{black}{).\ \ The
average
of\ }}\textrm{\textit{\textcolor{black}{m}}}\textrm{\textcolor{black}{\ }}\textrm{\textit{\textcolor{black}{n}}}\textrm{\textcolor{black}{*'s
was calculated for a set
of\ }}\textrm{\textit{\textcolor{black}{m}}}\textrm{\textcolor{black}{\ }}\textrm{\textit{\textcolor{black}{n}}}\textrm{\textcolor{black}{*'s.}}
\item \textrm{\textit{\textcolor{black}{s}}}\textrm{\textcolor{black}{\ (the number\ of\ the bootstrap rounds):\ it
represents how many times calculating the average
of\ }}\textrm{\textit{\textcolor{black}{m}}}\textrm{\textcolor{black}{\ }}\textrm{\textit{\textcolor{black}{n}}}\textrm{\textcolor{black}{*'s
is repeated for a given value of\ }}\textrm{\textit{\textcolor{black}{p}}}\textrm{\textcolor{black}{*.
\ We\ first\ randomly
selected\ }}\textrm{\textit{\textcolor{black}{m}}}\textrm{\textcolor{black}{\ }}\textrm{\textit{\textcolor{black}{n}}}\textrm{\textcolor{black}{*{}'s.
\ Then, we\ calculated the average
of\ the\ }}\textrm{\textit{\textcolor{black}{m}}}\textrm{\textcolor{black}{\ }}\textrm{\textit{\textcolor{black}{n}}}\textrm{\textcolor{black}{*{}'s.
\ Finally, we repeated this\ }}\textrm{\textit{\textcolor{black}{s}}}\textrm{\textcolor{black}{\ times to
calculate\ }}\textrm{\textit{\textcolor{black}{N}}}\textrm{\textcolor{black}{*.}}
\end{itemize}
\textrm{\textcolor{black}{Using\ }}\textrm{\textit{\textcolor{black}{p}}}\textrm{\textcolor{black}{*,\ }}\textrm{\textit{\textcolor{black}{k}}}\textrm{\textcolor{black}{,\ }}\textrm{\textit{\textcolor{black}{m}}}\textrm{\textcolor{black}{,\ }}\textrm{\textit{\textcolor{black}{s}}}\textrm{\textcolor{black}{,\ and\ }}\textrm{\textit{\textcolor{black}{${\theta}$}}}\textrm{\textcolor{black}{,\ the
following is the procedure of our method to
estimate\ }}\textrm{\textit{\textcolor{black}{N}}}\textrm{\textcolor{black}{* using the\ \ bootstrap
method.\ We\ first\ randomly\ generated\ }}\textrm{\textit{\textcolor{black}{k}}}\textrm{\textcolor{black}{\ sessions
of\ frequency attacks, which
produced\ }}\textrm{\textit{\textcolor{black}{k}}}\textrm{\textcolor{black}{\ }}\textrm{\textit{\textcolor{black}{n}}}\textrm{\textcolor{black}{*{}'s\ (i.e.,
each session of\ frequency attack\ generated a value
of\ }}\textrm{\textit{\textcolor{black}{n}}}\textrm{\textcolor{black}{*).\ Then, we randomly
selected\ }}\textrm{\textit{\textcolor{black}{m}}}\textrm{\textcolor{black}{\ }}\textrm{\textit{\textcolor{black}{n}}}\textrm{\textcolor{black}{*{}'s
from the set
of\ }}\textrm{\textit{\textcolor{black}{k}}}\textrm{\textcolor{black}{\ }}\textrm{\textit{\textcolor{black}{n}}}\textrm{\textcolor{black}{*{}'s.
\ From
the\ }}\textrm{\textit{\textcolor{black}{m}}}\textrm{\textcolor{black}{\ }}\textrm{\textit{\textcolor{black}{n}}}\textrm{\textcolor{black}{*'s,
we randomly
selected\ }}\textrm{\textit{\textcolor{black}{m}}}\textrm{\textcolor{black}{\ }}\textrm{\textit{\textcolor{black}{n}}}\textrm{\textcolor{black}{*'s,
by repeating the following procedure m times:}}

\textrm{\textcolor{black}{[F081?]\ selected\ one\ }}\textrm{\textit{\textcolor{black}{n}}}\textrm{\textcolor{black}{*'s
from the pool
of\ }}\textrm{\textit{\textcolor{black}{m}}}\textrm{\textcolor{black}{\ }}\textrm{\textit{\textcolor{black}{n*}}}\textrm{\textcolor{black}{{}'s}}

\textrm{\textcolor{black}{[F082?]\ put the selected\ }}\textrm{\textit{\textcolor{black}{n}}}\textrm{\textcolor{black}{*
back to the pool
of\ }}\textrm{\textit{\textcolor{black}{k}}}\textrm{\textcolor{black}{\ }}\textrm{\textit{\textcolor{black}{n}}}\textrm{\textcolor{black}{*'s}}

\textrm{\textcolor{black}{When\ a set
of\ }}\textrm{\textit{\textcolor{black}{m}}}\textrm{\textcolor{black}{\ }}\textrm{\textit{\textcolor{black}{n}}}\textrm{\textcolor{black}{*'s
were selected, we calculated the average of
the\ }}\textrm{\textit{\textcolor{black}{m}}}\textrm{\textcolor{black}{\ }}\textrm{\textit{\textcolor{black}{n}}}\textrm{\textcolor{black}{*{}'s.
\ We repeated the above\ }}\textrm{\textit{\textcolor{black}{s}}}\textrm{\textcolor{black}{\ times,
producing\ }}\textrm{\textit{\textcolor{black}{s}}}\textrm{\textcolor{black}{\ averages
of\ }}\textrm{\textit{\textcolor{black}{m}}}\textrm{\textcolor{black}{\ }}\textrm{\textit{\textcolor{black}{n}}}\textrm{\textcolor{black}{*'s.
This way, having
only\ }}\textrm{\textit{\textcolor{black}{m}}}\textrm{\textcolor{black}{\ (}}\textrm{\textit{\textcolor{black}{m}}}\textrm{\textcolor{black}{\ $<$$<$\ ${\infty}$)
samples
(i.e.,\ }}\textrm{\textit{\textcolor{black}{m}}}\textrm{\textcolor{black}{\ }}\textrm{\textit{\textcolor{black}{n}}}\textrm{\textcolor{black}{*'s),
we will be able to efficiently calculate\ }}\textrm{\textit{\textcolor{black}{N}}}\textrm{\textcolor{black}{* for a
large number of\ frequency attack\ sessions, improving the accuracy in
estimating\ }}\textrm{\textit{\textcolor{black}{N}}}\textrm{\textcolor{black}{* while it is tractable
(estimating\ }}\textrm{\textit{\textcolor{black}{N}}}\textrm{\textcolor{black}{*
from\ ${\infty}$\ }}\textrm{\textit{\textcolor{black}{n}}}\textrm{\textcolor{black}{*'s is not tractable).
\ Figure\ 3\ visualizes this concept.}}

\textrm{\textcolor{black}{To calculate\ }}\textrm{\textit{\textcolor{black}{N}}}\textrm{\textcolor{black}{* for a
certain confidence level, after we
calculated\ }}\textrm{\textit{\textcolor{black}{s}}}\textrm{\textcolor{black}{\ averages
of\ }}\textrm{\textit{\textcolor{black}{m}}}\textrm{\textcolor{black}{\ }}\textrm{\textit{\textcolor{black}{n}}}\textrm{\textcolor{black}{*'s,
by first identifying the bottom and the ceiling for the 95\% interval in the following way. \ We first ordered
the\ }}\textrm{\textit{\textcolor{black}{s}}}\textrm{\textcolor{black}{\ averages
of\ }}\textrm{\textit{\textcolor{black}{m}}}\textrm{\textcolor{black}{\ }}\textrm{\textit{\textcolor{black}{n}}}\textrm{\textcolor{black}{*'s
in the ascending order,
as\ }}\textrm{\textit{\textcolor{black}{n}}}\textrm{\textcolor{black}{*}}\textrm{\textcolor{black}{\textsubscript{[1]}}}\textrm{\textcolor{black}{\ through\ }}\textrm{\textit{\textcolor{black}{n}}}\textrm{\textcolor{black}{*}}\textrm{\textcolor{black}{\textsubscript{[}}}\textrm{\textit{\textcolor{black}{\textsubscript{s}}}}\textrm{\textcolor{black}{\textsubscript{]}}}\textrm{\textcolor{black}{.
\ Then, the\ bottom and the\ ceiling of a certain percentile were identified so that confidence intervals (we applied
95\%) were calculated by throwing out the\ bottom\ and\ top\ 2.5\% of its sampling distribution (Figure\ 4) as:}}

\textrm{\textcolor{black}{CI
=\ }}\textrm{\textit{\textcolor{black}{mean}}}\textrm{\textcolor{black}{\{}}\textrm{\textit{\textcolor{black}{N}}}\textrm{\textcolor{black}{*{\textbar}}}\textrm{\textit{\textcolor{black}{p*}}}\textrm{\textcolor{black}{\ =
0.36\}\ ${\pm}$\ 2\ ${\cdot}$\ SE\{}}\textrm{\textit{\textcolor{black}{N}}}\textrm{\textcolor{black}{*{\textbar}}}\textrm{\textit{\textcolor{black}{p*}}}\textrm{\textcolor{black}{\ =
0.36\}}}

{\rmfamily\color{black}
where ``SE'' means ``Standard Error''.}


\bigskip

{\centering   [Warning: Image ignored] % Unhandled or unsupported graphics:
%\includegraphics[width=3in,height=1.83333in]{mab-img003.emf}
 \par}
\textrm{\textbf{\textcolor{black}{Figure\ 3}}}\textrm{\textcolor{black}{:\ Bootstrapping
of\ }}\textrm{\textit{\textcolor{black}{m}}}\textrm{\textcolor{black}{\ samples}}

{\centering   [Warning: Image ignored] % Unhandled or unsupported graphics:
%\includegraphics[width=2.65625in,height=2.07292in]{mab-img004.emf}
 \par}
\textrm{\textbf{\textcolor{black}{Figure\ 4}}}\textrm{\textcolor{black}{: 95\% confidence interval
for\ the\ }}\textrm{\textit{\textcolor{black}{s}}}\textrm{\textcolor{black}{\ averages
of\ }}\textrm{\textit{\textcolor{black}{m}}}\textrm{\textcolor{black}{\ }}\textrm{\textit{\textcolor{black}{n}}}\textrm{\textcolor{black}{*{}'s}}

\textrm{\textcolor{black}{We defined a term
``}}\textrm{\textit{\textcolor{black}{${\theta}$}}}\textrm{\textcolor{black}{\ quantile'' (i.e., ``0.05 quantile'',
if\ }}\textrm{\textit{\textcolor{black}{${\theta}$}}}\textrm{\textcolor{black}{\ = 0.05), to mean
``(}}\textrm{\textit{\textcolor{black}{${\theta}$}}}\textrm{\textcolor{black}{\ ${\cdot}$\ 100) percentile'', which is
the\ }}\textrm{\textit{\textcolor{black}{n}}}\textrm{\textcolor{black}{*}}\textrm{\textcolor{black}{\textsubscript{[(}}}\textrm{\textit{\textcolor{black}{\textsubscript{${\theta}$\ ${\cdot}$\ k}}}}\textrm{\textcolor{black}{\textsubscript{)]}}}\textrm{\textcolor{black}{\ for
the actual distribution
and\ }}\textrm{\textit{\textcolor{black}{n}}}\textrm{\textcolor{black}{*}}\textrm{\textcolor{black}{\textsubscript{[(}}}\textrm{\textit{\textcolor{black}{\textsubscript{${\theta}$\ ${\cdot}$\ s}}}}\textrm{\textcolor{black}{\textsubscript{)]\ }}}\textrm{\textcolor{black}{for
the empirical distributions. The term, ``the actual distribution'' means the distribution of the average
of\ }}\textrm{\textit{\textcolor{black}{k}}}\textrm{\textcolor{black}{\ }}\textrm{\textit{\textcolor{black}{n}}}\textrm{\textcolor{black}{*'s,
while ``the empirical distribution'' means the one of
the\ }}\textrm{\textit{\textcolor{black}{s}}}\textrm{\textcolor{black}{\ averages
of\ }}\textrm{\textit{\textcolor{black}{m}}}\textrm{\textcolor{black}{\ }}\textrm{\textit{\textcolor{black}{n}}}\textrm{\textcolor{black}{*'s
(using our bootstrap method).\ We calculated the 95\% CI for each percentile we tested.}}

\textrm{\textcolor{black}{\ \ Finally, we calculated the estimated number of encrypted words that allow an adversary to
achieve an adversary accuracy of\ }}\textrm{\textit{\textcolor{black}{p}}}\textrm{\textcolor{black}{* with a
probability of\ }}\textrm{\textit{\textcolor{black}{${\theta}$}}}\textrm{\textcolor{black}{\ (i.e.,
``}}\textrm{\textit{\textcolor{black}{N}}}\textrm{\textcolor{black}{*'') using the following formula:}}

\textrm{\textit{\textcolor{black}{N*}}}\textrm{\textcolor{black}{\ =\ }}\textrm{\textit{\textcolor{black}{F}}}\textrm{\textcolor{black}{\textsuperscript{{}-1}}}\textrm{\textcolor{black}{(}}\textrm{\textit{\textcolor{black}{${\theta}$\ }}}\textrm{\textcolor{black}{{\textbar}\ }}\textrm{\textit{\textcolor{black}{p}}}\textrm{\textcolor{black}{*)}}

{\rmfamily\bfseries\color{black}
4. Performance Evaluation}


\bigskip

\textrm{\textcolor{black}{We\ compared\ the performance of our bootstrap method\ for
estimating\ }}\textrm{\textit{\textcolor{black}{N}}}\textrm{\textcolor{black}{*\ using
only\ }}\textrm{\textit{\textcolor{black}{m}}}\textrm{\textcolor{black}{\ samples\ with that of the ideal
case\ (i.e.,\ the one for representing\ the population set),\ using a large
value\ }}\textrm{\textit{\textcolor{black}{k}}}\textrm{\textcolor{black}{, which will
produce\ }}\textrm{\textit{\textcolor{black}{N}}}\textrm{\textcolor{black}{*\ close to\ that
for\ the\ ${\infty}$\ sample size.\ \ For that purpose, we compared the\ CDF's\ of\ the\ 5th percentile
(i.e.,\ }}\textrm{\textit{\textcolor{black}{${\theta}$}}}\textrm{\textcolor{black}{\ =\ 0.05) for the\ actual
distribution and\ the\ empirical distribution.}}

\textrm{\textcolor{black}{Figure 5 (a), (b), and (c) show the actual\ (the population set)\ and empirical CDF's from our
experiments for the following configuration:\ }}\textrm{\textit{\textcolor{black}{p}}}\textrm{\textcolor{black}{* =
0.30,\ }}\textrm{\textit{\textcolor{black}{${\theta}$}}}\textrm{\textcolor{black}{\ =
0.05,\ }}\textrm{\textit{\textcolor{black}{k}}}\textrm{\textcolor{black}{\ = 5,000,
and\ }}\textrm{\textit{\textcolor{black}{s}}}\textrm{\textcolor{black}{\ = 10,000
for\ }}\textrm{\textit{\textcolor{black}{m}}}\textrm{\textcolor{black}{\ = 50, 100, and 250, respectively. \ The
experiments demonstrated the following observations. \ For a small sampling size
(}}\textrm{\textit{\textcolor{black}{m}}}\textrm{\textcolor{black}{\ = 50), the actual and empirical sampling
distributions are nothing alike. \ As\ }}\textrm{\textit{\textcolor{black}{m}}}\textrm{\textcolor{black}{\ (the
bootstrap sampling size) increases, the actual and empirical sampling distributions quickly converge to the same shape.
\ At and after\ }}\textrm{\textit{\textcolor{black}{m}}}\textrm{\textcolor{black}{\ = 500, no significant difference
was observed between the two distributions.}}

{\centering  \includegraphics[width=2.76042in,height=2.0625in]{mab-img005.png} \par}
\textrm{\textbf{\textcolor{black}{Figure\ 5\ (a)}}}\textrm{\textcolor{black}{: The actual and empirical CDF's of\ the
5th percentile\ for\ }}\textrm{\textit{\textcolor{black}{m}}}\textrm{\textcolor{black}{\ (bootstrap\ sample size)\ =
50}}

{\centering  \includegraphics[width=2.76042in,height=2.08333in]{mab-img006.png} \par}
\textrm{\textbf{\textcolor{black}{Figure\ 5\ (b)}}}\textrm{\textcolor{black}{: The\ actual and empirical\ CDF's\ of the
5th percentile\ for\ }}\textrm{\textit{\textcolor{black}{m}}}\textrm{\textcolor{black}{\ = 100}}

{\centering  \includegraphics[width=2.83333in,height=2.04167in]{mab-img007.png} \par}
\textrm{\textbf{\textcolor{black}{Figure\ 5\ (c)}}}\textrm{\textcolor{black}{: The\ actual and empirical\ CDF's\ of the
5th percentile\ for\ }}\textrm{\textit{\textcolor{black}{m}}}\textrm{\textcolor{black}{\ = 250}}


\bigskip

\textrm{\textcolor{black}{We studied the effect
of\ }}\textrm{\textit{\textcolor{black}{${\theta}$}}}\textrm{\textcolor{black}{\ to\ }}\textrm{\textit{\textcolor{black}{N}}}\textrm{\textcolor{black}{*
when we increased\ }}\textrm{\textit{\textcolor{black}{${\theta}$}}}\textrm{\textcolor{black}{\ from 0.01 to 0.50.
\ Our simulation experiments for the above analyses generated approximately 6GB of raw data, from which we made the
following observations.}}

\textrm{\textcolor{black}{Figure 6 shows the\ }}\textrm{\textit{\textcolor{black}{N}}}\textrm{\textcolor{black}{*'s, as
well as their lower and the upper 95\% thresholds, for different levels of the probability
(}}\textrm{\textit{\textcolor{black}{${\theta}$}}}\textrm{\textcolor{black}{\ = 0.01 through 0.50) an
adversary\ achieves for an accuracy level of\ }}\textrm{\textit{\textcolor{black}{p}}}\textrm{\textcolor{black}{* =
0.30. \ The means between the bottom and ceiling of 95\% CI (the crosshairs on the solid line) were calculated by
taking the means of their sampling distribution.}}

\textrm{\textcolor{black}{We repeated the same analyses for the adversary accuracy of 45 and 50\%
(i.e.,\ }}\textrm{\textit{\textcolor{black}{p}}}\textrm{\textcolor{black}{* = 045 and 0.50). \ This is a scenario in
which an adversary can correctly map 45 and 50\% of what the adversary observed by a 5\% chance
(}}\textrm{\textit{\textcolor{black}{${\theta}$}}}\textrm{\textcolor{black}{\ = 0.05), respectively.}}

  [Warning: Image ignored] % Unhandled or unsupported graphics:
%\includegraphics[width=3in,height=2.07292in]{mab-img008.emf}
 

\textrm{\textbf{\textcolor{black}{Figure\ 6}}}\textrm{\textcolor{black}{:\ }}\textrm{\textit{\textcolor{black}{N}}}\textrm{\textcolor{black}{*
for different levels of the
probability\ for\ }}\textrm{\textit{\textcolor{black}{${\theta}$}}}\textrm{\textcolor{black}{\ =\ 0.01\ through
0.50\ an adversary achieves\ for\ an\ adversary\ accuracy
of\ }}\textrm{\textit{\textcolor{black}{p}}}\textrm{\textcolor{black}{*\ = 0.30.}}

  [Warning: Image ignored] % Unhandled or unsupported graphics:
%\includegraphics[width=3in,height=2.08333in]{mab-img009.emf}
 

\textrm{\textbf{\textcolor{black}{Figure\ 7}}}\textrm{\textcolor{black}{:\ }}\textrm{\textit{\textcolor{black}{N}}}\textrm{\textcolor{black}{*\ for
different levels of the probability for\ }}\textrm{\textit{\textcolor{black}{${\theta}$}}}\textrm{\textcolor{black}{\ =
0.01 through 0.50 an adversary achieves for an accuracy level
of\ }}\textrm{\textit{\textcolor{black}{p}}}\textrm{\textcolor{black}{* = 0.45 and 0.50.}}


\bigskip

\textrm{\textcolor{black}{Figure\ 7\ shows the\ }}\textrm{\textit{\textcolor{black}{N}}}\textrm{\textcolor{black}{*'s to
achieve\ }}\textrm{\textit{\textcolor{black}{p}}}\textrm{\textcolor{black}{* = 0.45 and 0.5 at the probability of 5\%
(}}\textrm{\textit{\textcolor{black}{${\theta}$}}}\textrm{\textcolor{black}{\ = 0.05). We estimated the mean is around
19,100 observations (}}\textrm{\textit{\textcolor{black}{N}}}\textrm{\textcolor{black}{* =
19,100)\ for\ }}\textrm{\textit{\textcolor{black}{p}}}\textrm{\textcolor{black}{* = 0.45.}}

{\rmfamily\color{black}
The followings are the observations from our experiments:}

\liststyleLFOxii
\begin{itemize}
\item \textrm{\textcolor{black}{The\ proposed MAB method calculated the\ estimated\ number of encrypted search queries
an adversary needs to observe\ (}}\textrm{\textit{\textcolor{black}{N}}}\textrm{\textcolor{black}{*)\ for\ achieving a
given accuracy level,\ }}\textrm{\textit{\textcolor{black}{p}}}\textrm{\textcolor{black}{*\ = 0.30, at the confidence
level of 95\%\ using only 5\% of the actual observations\ (250/5000)\ (Figure\ 5\ (c)).}}
\item \textrm{For 5\% chance that\ }\textrm{\textcolor{black}{the adversary can read 30\% of what the adversary observed
(i.e.,\ }}\textrm{\textit{\textcolor{black}{${\theta}$}}}\textrm{\textcolor{black}{\ = 0.05
for\ }}\textrm{\textit{\textcolor{black}{p}}}\textrm{\textcolor{black}{* = 0.3), we estimated that the adversary would
need around 6,900 observations (6,900 encrypted
words,\ }}\textrm{\textit{\textcolor{black}{N}}}\textrm{\textcolor{black}{* = 6,900). \ The 95\% confidence interval of
all the\ }}\textrm{\textit{\textcolor{black}{s}}}\textrm{\textcolor{black}{\ averages
of\ }}\textrm{\textit{\textcolor{black}{m}}}\textrm{\textcolor{black}{\ }}\textrm{\textit{\textcolor{black}{n}}}\textrm{\textcolor{black}{*{}'s
was [${\alpha}$:6,300,\ ${\beta}$:7,550]\ encrypted
words\ when\ }}\textrm{\textit{\textcolor{black}{m}}}\textrm{\textcolor{black}{\ = 500\ (Figure\ 6).}}
\item \textrm{\textcolor{black}{For comparison, for 30\% accuracy
(}}\textrm{\textit{\textcolor{black}{p}}}\textrm{\textcolor{black}{* = 0.30), an adversary only needs around
5,000\ encrypted words to be observed\ (}}\textrm{\textit{\textcolor{black}{N}}}\textrm{\textcolor{black}{* =
5,000)\ (Figure\ 6).}}
\item \textrm{\textcolor{black}{For a 50\% chance of success
(i.e.,\ }}\textrm{\textit{\textcolor{black}{${\theta}$}}}\textrm{\textcolor{black}{\ = 0.50), we observed that an
adversary would need around 17,000 observations\ (}}\textrm{\textit{\textcolor{black}{N}}}\textrm{\textcolor{black}{* =
17,000)\ for 50\% of success in achieving\ }}\textrm{\textit{\textcolor{black}{p}}}\textrm{\textcolor{black}{* =
0.30\ with\ }}\textrm{\textit{\textcolor{black}{m}}}\textrm{\textcolor{black}{\ = 500\ (Figure\ 7).}}
\item \textrm{\textcolor{black}{For 45\% accuracy
(i.e.,\ }}\textrm{\textit{\textcolor{black}{p}}}\textrm{\textcolor{black}{* = 0.45), we estimated that an adversary
would need a sample size around 360,000 (}}\textrm{\textit{\textcolor{black}{N*}}}\textrm{\textcolor{black}{\ =
360,000) to have a 1\% chance of success
(}}\textrm{\textit{\textcolor{black}{${\theta}$}}}\textrm{\textcolor{black}{\ = 0.01)\ (Figure\ 7).}}
\item \textrm{\textcolor{black}{For 50\% accuracy
(i.e.,\ }}\textrm{\textit{\textcolor{black}{p}}}\textrm{\textcolor{black}{* = 0.50), an adversary would need around 1.3
million samples (}}\textrm{\textit{\textcolor{black}{N}}}\textrm{\textcolor{black}{*\ =
1.3\ ${\cdot}$\ 10}}\textrm{\textcolor{black}{\textsuperscript{6}}}\textrm{\textcolor{black}{)\ (Figure\ 7).}}
\end{itemize}

\bigskip

{\rmfamily\bfseries
5.\ Conclusions and Future Work}


\bigskip

\textrm{T}\textrm{\textcolor{black}{he primary contributions in this paper are two-folds. \ First, there has not been
much work for studying how safe encrypted searches are against\ frequency attacks, which can be\ measured\ by a large
number of attackers for long period of\ time, possibly infinitely long.\ We provide studies on the resilience of
encrypted searches against\ frequency attacks from the view point of\ resilience engineering approach to enhance
security on encrypted searches. \ Resilience engineering is a new way of enhancing safety by precisely estimating the
level of possible threats to a system and feeding them back to adjusting or re-designing the system to maintain the
acceptable level of safety [17].}}

\textrm{\textcolor{black}{Our second contribution is development of a\ new\ method, Moving Average Bootstrap
(MAB)\ method,\ which\ efficiently and accurately calculates\ the estimator for the minimum number of encrypted
words\ (}}\textrm{\textit{\textcolor{black}{N}}}\textrm{\textcolor{black}{*)\ an adversary needs to achieve a given
accuracy level\ (}}\textrm{\textit{\textcolor{black}{p}}}\textrm{\textcolor{black}{*)\ with a certain level of
confidence\ as soon as a relatively small number of
samples\ (}}\textrm{\textit{\textcolor{black}{n}}}\textrm{\textcolor{black}{)\ (i.e., encrypted words) are submitted by
legitimate users. \ Thus, the MAB\ method\ will let the defenders calculate the estimator at an early stage without
waiting for a large number of queries submitted by legitimate users.\ Especially from the view point of
``tractability'', calculating the estimator using, not to mention an infinitely large number of encrypted words, a
large number of encrypted words takes time (waiting for a large number of encrypted words\ to be\ submitted) and huge
storage (storage space to hold\ the submitted encrypted words) is\ required.}}

\textrm{\textcolor{black}{Our proposed MAB method calculated the estimated number of encrypted search queries an
adversary needs to observe (}}\textrm{\textit{\textcolor{black}{N}}}\textrm{\textcolor{black}{*) for achieving a given
accuracy level,\ }}\textrm{\textit{\textcolor{black}{p}}}\textrm{\textcolor{black}{* = 0.30, at the confidence level of
95\% using only 5\% of the actual observations (250/5000) (Figure\ 5\ (c)).\ Assuming that the increase in the time an
adversary needs to achieve a certain\ }}\textrm{\textit{\textcolor{black}{p}}}\textrm{\textcolor{black}{* is
proportional to the ratio in the increase of the number of the encrypted words observed by an adversary
(}}\textrm{\textit{\textcolor{black}{n}}}\textrm{\textcolor{black}{) for a large number of encrypted words, the MAB
method would allow a defender to estimate\ }}\textrm{\textit{\textcolor{black}{N}}}\textrm{\textcolor{black}{* in 5\%
of time (without waiting for legitimate users to issue a large number of encrypted words).\ We are currently performing
analyses using higher\ }}\textrm{\textit{\textcolor{black}{p}}}\textrm{\textcolor{black}{* (0.55 through 0.80) for
different levels of confidence (90 to 98\%) for observing how they affect the performance of MAB method\ and\ for
observing\ if there is any pathological case for MAB method.}}


\bigskip

{\rmfamily\bfseries
Selected\ References}


\bigskip

\textrm{\textcolor{black}{[1]\ \ S. Subashini\ and\ V.\ Kavitha, ``A Survey on Security Issues in Service Delivery
Models of Cloud Computing,''\ }}\textrm{\textit{\textcolor{black}{Journal of Network\ and Computer
Applications}}}\textrm{\textcolor{black}{,\ vol. 34, no. 1, pp. 1-11, 2011.\ }}

\textrm{\textcolor{black}{[2]\ \ Dimitrios Zissis and Dimitrios Lekkas, ``Addressing Cloud Computing Security
Issues,''\ }}\textrm{\textit{\textcolor{black}{Future Generation Computer Systems}}}\textrm{\textcolor{black}{,\ vol.
28, no. 3, pp. 583-592, 2012.}}

\textrm{\textcolor{black}{[3]}}\textcolor{black}{\ \ }\textrm{\textcolor{black}{William R. Claycomb and Alex Nicoll,
``Insider Threats to Cloud Computing: Directions for New Research
Challenges,''\ }}\textrm{\textit{\textcolor{black}{Proceedings of\ IEEE\ International Conference on Computer Software
and Applications}}}\textrm{\textcolor{black}{, pp.\ 388-394,\ July\ 2012.}}

\textrm{\textcolor{black}{[4]\ \ Dan Boneh, Giovanni Di Crescenzo, Rafail Ostrovsky, and Giuseppe Persiano, ``Public Key
Encryption with Keyword Search,''\ }}\textrm{\textit{\textcolor{black}{Proceedings of
Eurocrypt}}}\textrm{\textcolor{black}{, pp. 506-522, May 2004.}}

\textrm{\textcolor{black}{[5]\ \ Jin Li, Qian Wang, Cong Wang, Ning Cao, Kui Ren, and Wenjing Lou, ``Fuzzy Keyword
Search over Encrypted Data in Cloud Computing,''\ }}\textrm{\textit{\textcolor{black}{Proceedings of IEEE
INFOCOM}}}\textrm{\textcolor{black}{, pp.\ 441-445,\ March\ 2010.}}

\textrm{\textcolor{black}{[6]\ \ Ning Cao, Cong Wang, Ming Li, Kui Ren, and Wenjing Lou, ``Privacy-Preserving
Multi-Keyword Ranked Search over Encrypted Cloud Data,''\ }}\textrm{\textit{\textcolor{black}{IEEE Transactions on
Parallel and Distributed Systems}}}\textrm{\textcolor{black}{, vol. 25, no. 1, pp. 222-233, January 2014.}}

\textrm{\textcolor{black}{[7] Wenhai Sun, Bing Wang, Ning Cao, Ming Li, Wenjing Lou, Thomas Y. Hou, and Hui Li,
``Privacy-Preserving Multi-Keyword Text Search in the Cloud Supporting Similarity-Based
Ranking,''\ }}\textrm{\textit{\textcolor{black}{Proceedings of the ACM Symposium on Information, Computer and
Communications Security}}}\textrm{\textcolor{black}{, pp. 71-82,\ May 2013.}}

\textrm{\textcolor{black}{[8]\ \ Seny Kamara and Kristin Lauter, ``Cryptographic Cloud
Storage,''}}\textcolor{black}{\ }\textrm{\textit{\textcolor{black}{Financial\ Cryptography and Data
Security}}}\textrm{\textcolor{black}{,\ }}\textrm{\textit{\textcolor{black}{Lecture Notes in Computer
Science}}}\textrm{\textcolor{black}{, vol.\ 6054,\ pp 136-149, 2010.}}

\textrm{\textcolor{black}{[9]\ \ Jean-Fran\c{c}ois Raymond, ``Traffic Analysis: Protocols, Attacks, Design Issues, and
Open Problems,''\ }}\textrm{\textit{\textcolor{black}{Designing Privacy Enhancing
Technologies}}}\textrm{\textcolor{black}{,\ }}\textrm{\textit{\textcolor{black}{Lecture Notes in Computer
Science}}}\textrm{\textcolor{black}{, vol. 2009, \ pp. 10-29, March 2001.}}

\textrm{\textcolor{black}{[10]\ \ Steven T. Piantadosi, ``Zipf's Word Frequency Law in Natural Language: A Critical
Review and Future Directions,''\ }}\textrm{\textit{\textcolor{black}{Psychonomic Bulletin \&
Review}}}\textrm{\textcolor{black}{, vol. 21, no. 5, pp. 1112-1130, October 2014.}}

\textrm{\textcolor{black}{[11]\ \ Jin Wook Byun, Hyun Suk Rhee, Hyun-A Park, and Dong Hoon Lee, ``Off-Line Keyword
Guessing Attacks on Recent Keyword Search Schemes over Encrypted Data,''\ }}\textrm{\textit{\textcolor{black}{Secure
Data Management}}}\textrm{\textcolor{black}{,\ }}\textrm{\textit{\textcolor{black}{Lecture Notes in Computer
Science}}}\textrm{\textcolor{black}{, vol. 4165, pp. 75-83, 2006.}}


\bigskip


\bigskip


\bigskip


\bigskip


\bigskip


\bigskip


\bigskip


\bigskip


\bigskip


\bigskip


\bigskip


\bigskip


\bigskip


\bigskip


\bigskip


\bigskip


\bigskip


\bigskip


\bigskip


\bigskip


\bigskip

{\rmfamily\color{black}
[12]\ \ Wei-Chuen\ Yau, Swee-Huay\ Heng, and Bok-Min\ Goi, ``Off-Line Keyword Guessing Attacks on Recent Public Key
Encryption with Keyword Search Schemes,'' Proceedings of the International Conference on Autonomic and Trusted
Computing, pp. 100--105, 2008.}

\textrm{\textcolor{black}{[13]\ \ Ik\ Rae Jeong, Jeong Ok Kwon, Dowon\ Hong, and Dong Hoon\ Lee. ``Constructing PEKS
Schemes Secure against Keyword Guessing Attacks is Possible?''\ }}\textrm{\textit{\textcolor{black}{Computer
Communications Express}}}\textrm{\textcolor{black}{, vol. 32, no. 2,\ pp. 394-396, 2009.}}

\textrm{\textcolor{black}{[14]\ \ Matthieu Rivain, ``On the Exact Success Rate of Side Channel Analysis in the Gaussian
Model,''\ }}\textrm{\textit{\textcolor{black}{Selected Areas in Cryptography}}}\textrm{\textcolor{black}{, vol. 5381,
pp. 165-183, 2009.}}

\textrm{\textcolor{black}{[15]\ \ Adrian Thillard, Emmanuel Prouff, and Thomas Roche, ``Success through Confidence:
Evaluating the Effectiveness of a Side-Channel Attack,''\ }}\textrm{\textit{\textcolor{black}{Cryptographic Hardware
and Embedded Systems}}}\textrm{\textcolor{black}{, vol. 8086, pp. 21-36, 2013.}}

\textrm{\textcolor{black}{[16]\ \ Debashis Kusharya, ``Bootstrap Methods and Their
Application,''\ }}\textrm{\textit{\textcolor{black}{Technometrics}}}\textrm{\textcolor{black}{, vol. 42, no. 2, pp.
216-217,\ March 2012.}}

\textrm{\textcolor{black}{[17]\ \ Resilience Engineering Association, ``About Resilience Engineering,''
URL:\ }}\textrm{\textit{\textcolor{black}{http://www.
resilience-engineering-association.org}}}\textrm{\textcolor{black}{/ (last\ accessed on February 16, 2015).}}
\end{document}
